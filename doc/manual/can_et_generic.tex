\chapter{Module CAN et g�n�ricit�}
\authors{
  \authorinfo{Julien}{Peeters} \\
  \authorinfo{Fabien}{Provost} \\
  \authorinfo{Feng}{Xiong} \\
  \authorinfo{Yongchao}{Xu} \\
  
}

\section{Description et contexte}
L'objecif de cette partie est de cr�er des outils permettant de simplifier l'utilisation des diff�rents bus de mani�re g�n�rique de fa�on � pouvoir �tre r�utilis� � plusieurs endroits du projet.

Seront vu dans cette partie, la conception et le d�veloppement des noeuds CAN qui permettent de recevoir des messages CAN et de les interpr�ter afin de g�n�rer les actions ad�quates (envoie de messages I�C par exemple) et inversement (envoie de message CAN apr�s utilisation d'un actionneur), aussi nous verrons l'utilisation d'un FPGA comme interface au bus I�C, c'est � dire que le bloc qui permet de connecter des DEL, des boutons poussoirs ou autre composants plus complexes � un bus I�C. 

\section{Noeud CAN/I�C}
\subsection{Objectifs}
L'objectif de ce module est donc double
\begin{itemize}
\item la recherche de modifications sur le bus I�C, puis envoie des changements par messages CAN
\item traiter les messages CAN re�us et envoie de messages I�C aux esclaves concern�s 

\end{itemize}



