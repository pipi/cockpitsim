\chapter{FSAdapter: Interface avec Microsoft Flight Simulator X (FSUIPC)}
\authors{
  \authorinfo{Julien}{Peeters} \\
  \authorinfo{Alexandra}{Starynkevitch} \\
  \authorinfo{Zhibin}{Yin}
}

\section{Description et contexte}
L'objectif de cette partie du projet est de cr�er un lien entre le simulateur physique
et le simulateur logiciel. Pour cela, nous disposions d�j� d'un acc�s aux donn�es
du mod�le a�ronautique via FSUIPC \footnote{Logiciel d�velopp� par Peter Dowson
(\url{http://www.schiratti.com/dowson.html}) permettant d'acc�der aux donn�es du
mod�le sous la forme d'un tableau index�.}.

Lors des d�bats technologiques initiaux, le choix s'est port� vers le protocole CAN
\footnote{Acronyme de \textit{Controller Area Network}.} pour g�rer l'�change de donn�es
entre la partie physique et la partie logicielle. Outre la gestion des donn�es via FSUIPC,
notre logiciel, FSAdapter, a �galement la charge de g�rer les messages qui transitent sur
le bus CAN, en r�ception ainsi qu'en �mission.

Sur base de cette architecture, illustr�e en \ref{fig:archi_fsadapter}, nous devions mettre
en place les points suivants :
\begin{enumerate}
  \item Offrir un acc�s facile et s�r aux donn�es du mod�le a�ronautique via FSUIPC.
  \item Ajouter une gestion des messages transitant sur le bus, ce qui inclu la mise
  � jour des donn�es du mod�le lors de la r�ception d'un message depuis le simulateur
  physique et l'�mission d'un message vers ce dernier lors de la d�tection d'un 
  changement dans le mod�le.
\end{enumerate}
