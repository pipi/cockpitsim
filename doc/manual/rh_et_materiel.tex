\chapter{Gestion ressources humaines et mat�riels}
\authors{
  \authorinfo{Romain}{Marchaudon} \\
}

\section{Responsabilit�s au sein du projet}

Lors de la cr�ation de l'�quipe projet, nous nous sommes tout de suite r�unis afin de d�cider des grands postes � pourvoir
au sein de ce groupe. Il a �t� d�cid� d'�lire un chef de projet ainsi un responsable technique ainsi
qu'un adjoint au management. C'est ainsi que nous avons nomm�:
\begin{itemize}
  \item Chef de projet: Marc de la Motte Rouge
  \item Responsable technique : Julien Peeters
  \item Adjoint au management : Romain Marchaudon
\end{itemize}

Ces personnes allaient alors �tre responsables � diff�rent niveau de la gestion de projet.

Marc, en tant que chef de projet, supervisera les taches, d�cidera de la gestion des �quipes et sera en charge
des choix technologiques pouvant �tre fait au cours du projet. Julien pour sa part aura en charge la gestion du
mat�riel, sera le repr�sentant des groupes de d�vellopement et pr�sentera les choix technologiques en accord avec les
�quipes de d�vellopement au chef de projet. Pour finir Romain, en tant qu'adjoint au management, s'occupera de la
communication avec les fournisseurs et les administrations qui ont la charge de la gestion du budget. Les commandes
fournisseurs seront pour leur part �tablies entre ces trois responsables en fonction des besoins en mat�riel des
diff�rentes �quipes.

\section{Gestion des �quipes de d�vellopement}

D'apr�s les diff�rents d�coupages qui ont �t� effectu�s, nous avons �clater l'�quipe en plusieurs entit�s
diff�rents correspondant � des groupes de d�vellopement.

C'est ainsi que 4 groupes de d�vellopement ont �t� d�finis:
\begin{itemize}
  \item Groupe FsAdapter
  \item Groupe FPGA
  \item Groupe BECK
  \item Groupe AX12
\end{itemize}

