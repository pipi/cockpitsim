% Introduction

\addcontentsline{toc}{chapter}{Introduction}

\chapter*{Introduction}
Dans le cadre de notre deuxi�me ann�e en �cole d'ing�nieur � Polytech Paris Sud en Electronique et Syt�me embarqu�, nous avonc r�alis� un cockpit d'avion. Notre projet �tait la cr�ation de la partie mat�riel et logiciel, avec le concept "hardware in the loop". En effet, nous avons recherch� le but p�dagogique et donc privil�gi� l'aspect �lectronique par rapport a l'aspect visuel.

Nous avons utilis� le logiciel Flight Simulator pour simuler le monde de l'aviation et obtenir des informations concernant l'avion et sont environnement. Nous avons ensuite manipul� ces donn�es, afin de pouvoir les visualiser et les modifier par une interface que nous avons cr�� pour mod�liser le cockpit.
	
Au cour de notre projet, nous avons divis� les t�ches � r�aliser en diff�rent blocs fonctionnels. Ceux-ci dialogue entre eux par diff�rents protocoles, comme CAN, I2C ou encore RS232. Le but de cette s�gmentation �tait de d�velopper des blogs g�n�rique, qui peuvent �tre r�utiliser pour cr�er d'autres entit�es dans le meme projet.

Nous nous sommes r�partis en diff�rents groupe de travail. Ces groupes ont �volu�s au cour du projet. cela a permis de d�velopper plusieurs blocs en parall�le, puis de les imbriquer ensemble par la suite.
	