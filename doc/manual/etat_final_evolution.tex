\chapter{Etat final et possibilit�es d'�volutions}
\authors{
  \authorinfo{pr�nom}{nom} \\
}

\section{�volution de FSadapter}
�tant donn� que tout le materiel n�cessaire n'�tait pas disponible, nous avons d�cid� de prouver le concept de la r�alisation du cockpit, en interfa�ant un bouton permettant d'incr�menter la valeur de l'altitude. Les fonctions permettant d'utiliser cette valeur ont donc �t� �crites. Par contre, les autres fonctionnalit�es n'ont pas �t� ajout�es au simulateur. Autrement dit, pour chaque r�alisation mat�rielle faite sur le cockpit, des adaptations doivent �tre r�alis�es dans le programme. Les trames doivent ainsi �tre r�alis�es pour toutes les donn�es qui sont ajout�es au simulateur, et les fonctions de traitement (dans le sens CAN vers FS et FS vers CAN) doivent �tre �crites pour chaque famille de donn�es, et (le plus important) les donn�es obtenues par le bus CAN doivent �tre reli�es aux valeurs fournies par FSUIPC.
